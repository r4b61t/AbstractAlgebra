\documentclass[]{article}
\title{Assignment 2}
\author{Chayapon Thunsetkul 6280742}
\usepackage{amssymb,amsmath,mathtools,amsfonts,enumitem,tasks,physics,mathrsfs}
\usepackage[margin=1in]{geometry}
\newcommand\Label[1]{&\refstepcounter{equation}(\theequation)\ltx@label{#1}&}
\newcommand\numberthis{\addtocounter{equation}{1}\tag{\theequation}}
\usepackage{listings}
\usepackage{xcolor}
\usepackage{caption}
\usepackage{float}
\definecolor{dkgreen}{rgb}{0,0.6,0}
\definecolor{gray}{rgb}{0.5,0.5,0.5}
\definecolor{mauve}{rgb}{0.58,0,0.82}
\definecolor{backcolour}{rgb}{0.95,0.95,0.92}
\lstset{frame=tb, language=Java, aboveskip=3mm, belowskip=3mm, showstringspaces=false, columns=flexible, basicstyle={\small	tfamily}, numbers=left, numberstyle=\color{gray}, keywordstyle=\color{blue}, commentstyle=\color{dkgreen}, stringstyle=\color{mauve}, breaklines=true, breakatwhitespace=true, showtabs=false, tabsize=3 }
\newcommand{\R}{\mathbb{R}}
\begin{document}
\maketitle
\begin{description}
    \item[Question 1.] We know that $a+b \in \mathbb{R}$. Obviously, $a+b-1$ is also in $\mathbb{R}$. This satifies the closure property. Next, let $a,b,c \in \mathbb{R}$ then, $a * (b*c) = a + (b+c -1) -1 = a+ b + c -1 -1 = a+b-1 + c -1 = (a*b)*c$, hence, the associative property is satisfied. The identity of this operation is 1 since for all $a\in \mathbb{R}$ we have $1 * a = 1+ a - 1 = a = a+ 1 - 1 = a*1$. The inverse of $a \in \R$ is $-a+2 \in \R$ because $(-a+2)*a = -a+2 + a -1 = 1 = a - a + 2 - 1 = a * (-a+2)$. Lastly, to show that the operation $*$ is commutative, for all $a,b \in \R$, $a*b = a + b -1 = b + a -1 = b*a$. Thus, $\{G,*\}$ is an abelian group.
    \item[Question 2.] The closure and associative properties follow from matrix multiplication. The identity matrix, $\left[\begin{matrix}
        1&0\\0&1
    \end{matrix} \right]$ ,is clearly in the group. Next, for any  \[
        A  = \begin{bmatrix}
            m & b \\ 0&1
        \end{bmatrix}
    \]  in the group, we have its multiplicative inverse \[
    A^{-1}  =
        \frac{1}{m}  
\begin{bmatrix}
    1 & -b \\ 0 & m
    \end{bmatrix} =
\begin{bmatrix}
    1/m & -b/m \\ 0 & 1
    \end{bmatrix} 
    \] which is also an element of the group since $m\neq 0$
    \item[Question 3.] Let $a,b \in G$, since $b \neq 1$ we must have $\ln(b)\neq 0$. Next, since $a > 0$, it follows that $a^{\ln(b)} > 0$ and $ a\neq 1$. This satisfies the closure property. Now, for any $c \in G$, we have $a*(b*c) = a^{\ln(b^{\ln(c)})} =  a^{\ln(c)\ln(b)}= a^{\ln(c^{\ln(b)})} = (a*b)*c$. This satifies the associative property.
    Let $i \in \R$ such that $i*a = i^{\ln(a) }=a$ and $a*i = a^{\ln(i) }=a$, then, it follows that \begin{align*}
        \ln(a)\ln(i) &= \ln(a)\\
        i &= e
    \end{align*} Clearly, $i = e \in G$. Now, suppose we have $a^{-1} \in \R$ such that $a*a^{-1} = a^{-1}*a = e$. Then, \begin{align*}
        \ln(a){\ln(a^{-1})} &= \ln(e)\\
        a^{-1}&= e^{\frac{1}{\ln(a)}}
    \end{align*} and for all $a \in G$, $a^{-1} \neq 1, a^{-1} >0$, thus satifying as the inverse of $a$ in the group. Then, the inverse of $e^2$ is $e^{\frac{1}{\ln(e^2)}} = e^{1/2}$ 
    \item[Question 4.] The inverse of each element in $U(7)$ and the Cayley table for the group $U(7)$ is shown in table 1 and 2.\\
\begin{minipage}[c]{0.5\textwidth}
\centering
\begin{tabular}{|c|c|}
\hline
elements in $U(7)$ & Inverse \\
\hline
 1 & 1 \\
\hline 
 2 & 4 \\
\hline
 3 & 5 \\
\hline
 4 & 2 \\
\hline 
 5 & 3 \\
\hline
 6 & 6 \\
\hline
\end{tabular}
\captionof{table}{Inverse of elements in $U(7)$}
\end{minipage}
\begin{minipage}[c]{0.5\textwidth}
\centering
\begin{tabular}{c|cccccc}
    &1&2&3&4&5&6\\
    \hline
    1 &1&2&3&4&5&6\\
    2 &2&4&6&1&3&5\\
    3 &3&6&2&5&1&4\\
    4 &4&1&5&2&6&3\\
    5 &5&3&1&6&4&2\\
    6 &6&5&4&3&2&1
\end{tabular}
\captionof{table}{Cayley table for group $U(7)$}
\end{minipage}
\item[Question 5. ] Obviously, the identity matrix is an element of $H$. Next, for any diagonal matrix $A,B \in H$ we have its product; \[
 A \cdot B = 
 \begin{bmatrix}
    a_1 & 0 & \dots & 0\\
    0 & a_2 && \vdots\\
    \vdots && \ddots & 0 \\
    0 & \dots & 0 & a_n
 \end{bmatrix}  
 \cdot
 \begin{bmatrix}
    b_1 & 0 & \dots & 0\\
    0 & b_2 && \vdots\\
    \vdots && \ddots & 0 \\
    0 & \dots & 0 & b_n
 \end{bmatrix}  
 =
 \begin{bmatrix}
    a_1b_1 & 0 & \dots & 0\\
    0 & a_2b_2 && \vdots\\
    \vdots && \ddots & 0 \\
    0 & \dots & 0 & a_2b_n
 \end{bmatrix}  
\] is a diagonal matrix as well. Hence, the operation is closed within $H$. Lastly, each elements have a non-zero determinant. It follows that for every elements in $H$ are invertible. Furthermore, inverse of a diagonal matrix is also a diagonal matrix. Thus, for every element $h$ in $H$, its inverse $h^{-1}$ also exists within $H$. We conclude that $H \leqslant G$.
\item[Question 6.] Obviously, we have the identity $e \in G$ satifying $e^2 = e$, then $e \in H$ and $H$ is not empty. Now, let $h,g \in H$, we have \begin{align*}
    hg &= f \in G\\
    f^2 &= hghg \\
    f^2 &= hggh \\
    f^2 &= heh\\
    f^2 &= hh = e
\end{align*} Thus, $hg = f \in H$ and the operation is closed in $H$. Lastly, for any $g \in H$, $gg = e$ so clearly the inverse of $g$ is $g$ itself.
    

When $G = U(7)$, we have $H = \{1,6\}$ and when $G = U(8)$, we have $G = \{1,3,5\}$.

\item[Question 7.] Let $H = \{a \in G : 2a =0\}$. We can rewrite this in multiplicative notation to get $H = \{a \in G : a^2 = e\}$ then the proof of this can be found in the previous question.

When $G = \mathbb{Z}_{12}$, we have $H = \{0,6\}$. When $G = \mathbb{Z}_{13}$, we have $H = \{0\}$
\item[Quesiton 8.] The identity, $e \in G$ satifies $ ea = ae = a$ hence $e \in C(a)$. Next, let $x,y \in C(a)$ and their product, $xy = z \in G$. Now, $za = xya = xay = axy = az$ so $z \in C(a)$. Lastly, for any $x \in C(a)$ we have its inverse $x^{-1} \in G$ such that \begin{align*}
    x^{-1} x &= e\\
    x^{-1} x a&= a\\
    x^{-1} ax&= a\\
    x^{-1} axx^{-1}&= ax^{-1}\\
    x^{-1} a&= ax^{-1}
\end{align*}So $x^{-1} \in C(a)$.
\end{description}
\end{document}